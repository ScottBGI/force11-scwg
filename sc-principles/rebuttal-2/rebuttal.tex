\pdfoutput=1
\pdfminorversion=4
\documentclass{article}
\usepackage[left=1in,right=1in,top=1in,bottom=1in]{geometry}
\geometry{letterpaper}
\usepackage{graphicx, caption}
\usepackage{latexsym,url,amscd, amssymb, amsmath}
\usepackage{epstopdf}

\usepackage{multicol}
\usepackage{booktabs}
\usepackage{fancyvrb}

\usepackage[version=3]{mhchem} % Formula subscripts using \ce{}, e.g., \ce{H2SO4}

\usepackage[english]{babel}
\usepackage{textcomp}

\usepackage{siunitx}
\sisetup{group-separator={,},
         detect-all,
         binary-units,
         list-units = single,
         range-units = single,
         tophrase = --,
         per-mode = symbol-or-fraction,
         separate-uncertainty = true
%        scientific-notation = fixed
}
\DeclareSIUnit\atm{atm}
\DeclareSIUnit\CA{CA}
\DeclareSIUnit\rpm{rpm}

%to highlight text
\usepackage{soul}
\usepackage[usenames, dvipsnames]{xcolor}
\newcommand{\hly}[1]{{\sethlcolor{yellow}\hl{#1}}}
\newcommand{\hlb}[1]{{\sethlcolor{SkyBlue}\hl{#1}}}
\newcommand{\hlg}[1]{{\sethlcolor{green}\hl{#1}}}

\usepackage[sort&compress]{natbib}
\bibpunct{[}{]}{,}{n}{}{;}
\usepackage[linkcolor=blue, citecolor=blue, colorlinks=true, breaklinks=true]{hyperref}

\usepackage{url,breakurl}
\usepackage{enumitem}

\newcommand{\katznote}[1]{ {\textcolor{blue} { ***DSK: #1 }}} % Dan
\newcommand{\niemnote}[1]{ {\textcolor{orange} { ***KEN: #1 }}} % Kyle
\newcommand{\asnote}[1]{ {\textcolor{red} { ***AMS: #1 }}} %Arfon


%\date{}                                           % Activate to display a given date or no date

\setlength{\parindent}{0in}
\setlength{\parskip}{0.4cm}

%%% BEGIN DOCUMENT
\begin{document}

We thank the editor and reviewers for their helpful comments, and have addressed their requests and questions here.
Specific changes or additions can be found in the marked revision.
% in response to the Editor, Reviewer \#1, and Reviewer \#2 are respectively highlighted in the marked manuscript in \hly{yellow}, \hlb{blue}, and \hlg{green}.

%%%%%%%%%%%%%%%%%%%%%%%%%%%%

\section*{Editor:}

{\itshape
First of all, I would like to thank the authors for having addressed all my
comments and reviewers' suggestions in their revision (and clearly explained
their modifications in their response letter), and thanks again to the
reviewers for their wonderful job.

While the paper is now in a good shape for being published, there are just
minor issues that should be properly addressed before having it accepted
for the final publication. Such issues are highlighted by the reviewer, and
concern the new parts the authors have added, in particular:

\begin{enumerate}
\item the comparison between software and data;
\item the assignment of an identifier for explicit versions of a software by
means of repository capabilities (e.g. hashes).
\end{enumerate}

This is not a huge amount of work to add, as far as I can see, but a clear
discussion (even supported by appropriate references) would be beneficial
for the article.

Thanks again for your wonderful work.
}

\textbf{Our response:}

%%%%%%%%%%%%%%%%%%%%%%%%%%%%

\section*{Reviewer \#1 (Tobias Kuhn):}

{\itshape
The authors followed most of my suggestions, and where they did not, they
mostly presented convincing arguments. I like in particular that there is
now an appendix with the use cases.

However, there are two aspects of the latest version that I find not very
convincing:
}

\begin{enumerate}

\item \textbf{Reviewer comment:}
\emph{The listed differences between software and data (lines 100--114) are not
convincing in my opinion:}

\begin{itemize}
{\itshape

\item ``Software can be used to express or explain concepts, unlike
data'': I don't understand this.

\item ``Software is updated more frequently that papers or data''
(typo: ``that'' instead of ``than''): This was questioned also by the
second reviewer, and I don't think the issue is resolved. The footnote is
unsupported and seems to partly contradict the main point. (I can imagine that
maybe only *popular* software tends to be updated frequently, whereas
*popular* dataset are often not.)

\item ``Software suffers from a different type of bit rot than data'':
I am not convinced that this is really a different *type* of bit rot.
In particular datasets in proprietary formats seem to suffer from exactly the
same problems.

\item ``Software is frequently built to use other software'': True,
though similar things can occur in datasets too, in particular with respect to
Linked Data, where datasets typically use a number of third-party ontologies and
vocabularies.

\item ``Software is generally smaller than data'': Another unsupported
claim, and I am not fully convinced that this is true. There must be a large
number of very small spreadsheets of data. So, on average data might be smaller
than software. The difference is certainly true on the upper end of the scale
though: the largest datasets are larger than the largest pieces of software.

\item ``Software teams can be large and multidisciplinary, \ldots'': I think
this applies to data too.

\item ``The lifetime of software is generally not as long as that of data.'':
Again unsupported, and again I can imagine this to be false. Many datasets, in
particular small ones, might be stillbirths (i.e.\ they are never used). The
same might be true for software, but possibly to a lesser degree. And software
might commonly evolve, whereas data might more often be replaced by a
different and newer dataset, in which case evolving software would live
longer.
}

\end{itemize}

\textbf{Our response:} We, and the overall working group, believe there are significant difference
between software and data, and that these lead to differences in the corresponding citation principles
as well as their application.
However, we agree that there has not been sufficient discussion to document these
differences in a concise, well-structured, and referenced manner.
For this reason, we have removed the list of differences in the paper and point
to an ongoing discussion where a large set of community members are working
to complete this structured documentation with evidence of each point.  An example
of one difference is that appropriate licenses for software are different than appropriate
licenses for data (i.e., for open software, an OSI-approved license, while for open data, a CC license).


\item \textbf{Reviewer comment:}
\emph{I am also not convinced by the arguments against hashes as identifiers:}

\textbf{Our response:}
As a general comment, we prefer to recommend DOIs associated with specific,
released versions of software for a number of reasons: not specific to VCS used,
more guaranteed persistence, same persistent identifier used for publications and data.
However, there are certain sitatuions where this may not be possible:
(1) the software developers did not register a DOI or release specific
versions, (2) the version of software used does not match what is available
to cite (perhaps they used the dev/prerelease version), etc. In those cases,
falling back on URL+hash would be an approriate way to cite the software used.

Furthermore, one of the goals behind creating the software citation principles
is increasing the prominence of software as an equal research product to
publications and data, which both use DOIs as a standard identifier and locator.
Thus, we feel it is important (in combination with our other stated reasons) to
recommend a similar standard for software; while other solutions (such as URL +
hash) certainly exist, suggesting a different---if equivalently suitable---standard
would serve to propagate the idea that software is a lesser research output.

We would also like to draw the reviewer's attention to the fact that this paper
describes the result of the inital Software Citation Principles working group. A
followup effort around the implementation of these principles (in a variety of
settings) will likely provide more detailed guidance and discussion around what
exactly a software citation should look like and the data it should capture such
as a Git SHA/REF.

We have updated the manuscript to better match this position.

\begin{itemize}

\item \emph{``commit references are not guaranteed to be permanent'':
identifiers consisting of strong hash values *are* guaranteed to be permanent
in the sense that they can forever only represent what they were meant to
represent, i.e. the bits and bytes on which the hash was calculated (which, in
the case of Git, is the entire version history).}

\textbf{Our response:}
It is possible to overwrite\slash clobber a particular commit (e.g, using a
git force-push), or remove a repository entirely, and so access to that hash
may not be persistent.

\item \emph{``A repository address and version number does not guarantee that
the software is available at a particular (resolvable) URL''. It depends. A URL
like \url{https://github.com/torvalds/linux/commit/a0cba2179ea4c1820fce2ee046b6ed90ecc56196}
*does* come with a hash and can be resolved to get to the actual software.
There is no guarantee that github.com will be available forever, but this is
also true for publisher sites and therefore DOIs.}

\textbf{Our response:}
We agree that a repository URL combined with hash can be used to get to
the specific version of software referenced. However, this approach to citation
applies only to Git, which not all research software uses. As a result, multiple
citation solutions would be needed. We prefer to recommend a practice that is
independent of the software development tools used---archiving a software
version and registering a DOI does not require the use of Git, or in fact any
version control system at all (though we of course recommend it).

Regarding the peristence of GitHub vs.\ the DOI system, DOIs are an ISO standard
governed by a nonprofit foundation with a specific goal of persistence. Beyond
that, there are robust disaster recovery solutions in place for archivers and
publishers---the persistence of scholarly objects (and the DOI system) is
more certain than that of GitHub.

\item \emph{``A particular version number/commit reference may not represent a
``preferred'' point at which to cite the software from the perspective of the
package authors.'': Hashes certainly don't prevent you from defining preferred
versions for future reference. But they come with the *possibility* to identify
all versions, which I would in fact consider an important feature. If I want to
make my research perfectly reproducible, I should be able to cite the exact
version that I used, whether that happens to be a preferred version or not.}

\textbf{Our response:}
As stated in our general response above and reflected in our changes to the
manuscript, we agree hashes (combined with a
repository URL) could be used when citing a version of software that does not
have an associated DOI.
We do expect the follow-on implementation working group to work further on these solutions.


\item \emph{``When we talk about a ``Unique identifier'' in the document, we
are referring to unique, persistent, and machine-actionable identifiers such as
a DOI, ARK, or PURL; we do not believe a repository address with a version
number is sufficient.'' [from the rebuttal letter]: Identifiers that include
hashes are in important ways *more* unique, *more* persistent, and *more*
machine-actionable than DOIs or PURLs. Strong hashing algorithms guarantee
uniqueness (no other input can be constructed for the same hash), are in
practice probably more persistent (for example, there exist independent
complete copies of all GitHub content, so even if github.com closes for good,
the cited software could still be found), and can be automatically verified
against their supposed content. (The above holds for *strong* hash functions,
and Git's SHA-1 is no longer considered strong - even though not a single
collision has been found until now - but this doesn't invalidate the general
points.)}

\textbf{Our response:}
The working group focused on DOIs as the primary unique identifier\slash
locator to satisfy the principles of unique identification, persistence, and
accessibility. While strong hashing algorithms (which Git does not use) are
unique, they cannot currently be used with GitHub repo URLs (e.g.). In addition,
we prefer a persistent solution that doesn't depend on outside solutions if
(e.g.) GitHub goes away---the DOI system ensures persistence regardless of the
original ``owner''\slash naming authority.

\end{itemize}

\end{enumerate}

%\bibliography{}
%\bibliographystyle{}

\end{document}
